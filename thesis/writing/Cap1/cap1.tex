\section{Motivation}
Turbines play an important role when it comes to propulsion systems for both aeronautical and space and also power generation systems. Therefore, several countries including their thousands of companies, scientists and engineers have already dedicated lots of financial, temporal and computational resources to the development of turbines, seeking above all to improve their efficiency.

It is know that any gain in the efficiency of a gas turbine has direct effects on the reduction of fuel consumptioon and consequent financial savings, mainly when comes to axial flow turbines as they are present in several applications and have an exceptional high scalability. Such benefit can also be pursued and achieved if turbine preliminary design tools were more efficient and speeded up the conceptual phase of the its development. Then, there is a demand and need for tools of this type. 

\section{Thesis Structure}
The chapter 1 aims to introduce the project, where are exposed a brief identification of the problem, overall scene and the motivation to solve it, in order to support and justify the need for this work.

In the following, chapter 2 puts in the general and specific objectives, theme identification and contributions of the present work.

Chapter 3 presents the literature review on two fronts, along the presentation of the nomenclature used. The first is the physics that describes the axial flow turbines and its design, which basically includes concepts of fluid and thermodynamics, based on a reduced order methodology. Then, the second presents a description of the genetic algorithms and their characteristics that try to mimic the concepts of natural and genetic evolution, as a tool to find optimal solutions of the problem.

Chapter 4 brings the methodology used for the development of the computational tool, with the definition of the optimization process. 

Chapter 5 provides the results obtained from simulations, considering a turbine XX as a validation case. Then, an axial flow turbine is designed as beeing the optimal solution of the problem descripted in the previous chapter. Discussion and literature comparison also compoud this chapter, along the critical evaluation of the tool performance capability.

Chapter 6 and last delivers final consideration and proposes future work of the studied theme.
